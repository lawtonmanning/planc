\documentclass[conference,compsoc]{IEEEtran}
\IEEEoverridecommandlockouts
% The preceding line is only needed to identify funding in the first footnote. If that is unneeded, please comment it out.

\usepackage{cite,doi}
\usepackage{amsmath,amssymb,amsfonts}
\usepackage{algorithm,algpseudocode}
\usepackage{graphicx}
\usepackage{textcomp}
\usepackage{xcolor}
\usepackage{cleveref}
\usepackage{ourmacros}
\def\BibTeX{{\rm B\kern-.05em{\sc i\kern-.025em b}\kern-.08em
    T\kern-.1667em\lower.7ex\hbox{E}\kern-.125emX}}

\newcommand{\LM}[1]{\textcolor{blue}{\textbf{Lawton}: #1}}    
\newcommand{\GB}[1]{\textcolor{red}{\textbf{Grey}: #1}}
\newcommand{\RK}[1]{\textcolor{blue}{\textbf{Ramki}: #1}}


\begin{document}

\title{Parallel Hierarchical Clustering using \\ Rank-Two Nonnegative Matrix Factorization
\thanks{This material is based upon work supported by the National Science Foundation under Grant No. OAC-1642385 and OAC-1642410.
This manuscript has been co-authored by UT-Battelle, LLC under Contract No. DE-AC05-00OR22725 with the U.S. Department of
Energy. This project was partially funded by the Laboratory Director's Research and Development fund. This research used resources
of the Oak Ridge Leadership Computing Facility at the Oak Ridge National Laboratory, which is supported by the Office of Science of
the U.S. Department of Energy.}
}

\author{\IEEEauthorblockN{Lawton Manning and Grey Ballard}
\IEEEauthorblockA{Wake Forest University \\
Winston-Salem, NC, USA \\
\{mannlg15,ballard\}@wfu.edu}
\and
\IEEEauthorblockN{Ramakrishnan Kannan}
\IEEEauthorblockA{Oak Ridge National Laboratory \\
Oak Ridge, TN, USA \\
kannanr@ornl.gov}
\and
\IEEEauthorblockN{Haesun Park}
\IEEEauthorblockA{Georgia Institute of Technology \\
Atlanta, GA, USA \\
hpark@cc.gatech.edu}
}

\maketitle

\begin{abstract}
\GB{to be done...}
\end{abstract}

%\begin{IEEEkeywords}
%component, formatting, style, styling, insert
%\end{IEEEkeywords}

\section{Introduction}

test reference: \cite{GKP15}

\section{Preliminaries}

\section{Algorithm}

\subsection{Sequential Algorithm}

\subsubsection{Rank-2 NMF}

\subsubsection{Hierarchical Clustering}

\subsection{Parallelization}

\subsubsection{Algorithm}

\begin{algorithm}
\caption{TT-right-orthogonalization}
\label{alg:parrank2nmf}
\begin{algorithmic}[1]
	\Require{}
	\Ensure{}
	\Function{$\M{Y} =$ Parallel-Rank2-NMF}{$\M{A}$}
		\State
	\EndFunction
\end{algorithmic}
\end{algorithm}

\subsubsection{Analysis}

\section{Experimental Results}

\subsection{Experimental Platform}

\subsection{Datasets}

\subsection{Performance}

\subsubsection{Weak Scaling}

\subsubsection{Strong Scaling}

\subsection{Clustering}

\subsubsection{Hyperspectral Imaging}

\subsubsection{Document Corpus}

\section{Conclusion}

\section*{Acknowledgment}

The authors would like to thank Simin Ma for contributions to the algorithmic analysis and John Farrell for his contributions to the implementation of the parallel algorithm.

\bibliography{paper}
\bibliographystyle{plainurl}

\end{document}
