\documentclass{beamer}

\usepackage{cite,doi}
\usepackage{amsmath,amssymb,amsfonts}
\usepackage{algorithm,algpseudocode}
\usepackage{stackengine,graphicx}
\usepackage{textcomp}
\usepackage{xcolor}
\usepackage{cleveref}
\usepackage{tikz}
\usetikzlibrary{3d}
\usetikzlibrary{patterns}
\usetikzlibrary{calc}
\usetikzlibrary{arrows}
\usetikzlibrary{matrix}
\usetikzlibrary{positioning}
\usetikzlibrary{decorations.pathreplacing}
\usepackage{mathtools}
\usepackage{subcaption}
\usepackage{import}
\usepackage{bm}
\usepackage[mathscr]{eucal}
\usepackage{amsbsy}
\usepackage[section]{placeins}
\usepackage{adjustbox}
\stackMath

\usepackage{../ourmacros}

\newcommand{\T}[1]{\boldsymbol{\mathscr{#1}}}

\newcommand{\GB}[1]{\textcolor{red}{\textbf{GB}: #1}}
\newcommand{\hyper}{DC-HYDICE}
\newcommand{\image}{SIIM-ISIC}

\definecolor{wfugold}{rgb}{0.6196078,0.494117647,0.21960784}
\newcommand{\red}[1]{\textcolor{red}{#1}}
\newcommand{\blue}[1]{\textcolor{blue}{#1}}
\newcommand{\multiplycolor}{red}
\newcommand{\zero}{}
\newcommand{\cred}[1]{\textcolor{red}{#1}}
\newcommand{\cblue}[1]{\textcolor{blue}{#1}}
\newcommand{\cgold}[1]{\textcolor{wfugold}{#1}}
\newcommand{\email}[1]{\href{mailto:#1}{\texttt{#1}}}


\newlength\matfield
\newlength\tmplength
\def\matscale{1.}
\newcommand{\dimbox}[3]{%
  \setlength\matfield{\matscale\baselineskip}%
  \setbox0=\hbox{\vphantom{X}\smash{#3}}%
  \setlength{\tmplength}{#1\matfield-\ht0-\dp0}%
  \fboxrule=1pt\fboxsep=-\fboxrule\relax%
  \fbox{\makebox[#2\matfield]{\addstackgap[.5\tmplength]{\box0}}}%
}
\newcommand{\raiserows}[2]{%
   \setlength\matfield{\matscale\baselineskip}%
   \raisebox{#1\matfield}{#2}%
}
\newcommand{\matbox}[5]{
  \stackunder{\dimbox{#1}{#2}{$#5$}}{\scriptstyle(#3\times #4)}%
}

\title{Parallel Hierarchical Clustering using \\ Rank-Two Nonnegative Matrix Factorization}

\author{
    Lawton Manning\inst{1}
    \and Grey Ballard\inst{1}\\
    \and Ramakrishnan Kannan\inst{2}
    \and Haesun Park\inst{3}
}

\institute{
    \inst{1}%
    Wake Forest University
    \and
    \inst{2}%
    Oak Ridge National Laboratory
    \and
    \inst{3}%
    Georgia Institute of Technology
}

\date{
    27th IEEE International Conference on High Performance Computing, Data, \& Analytics (HiPC 2020)
}

\usetheme{Warsaw}
\usecolortheme[named=wfugold]{structure}
%Madrid
%\usecolortheme{dolphin}
%\useinnertheme{rounded}
%\usefonttheme{serif}
\setbeamertemplate{navigation symbols}{} % gets rid of navigation bars
\setbeamertemplate{footline}
{
  \hbox{%
  \begin{beamercolorbox}[wd=.33\paperwidth,ht=2.25ex,dp=1ex,left]{author in head/foot}%
    \usebeamerfont{author in head/foot}
    Manning, Ballard, Kannan, Park
  \end{beamercolorbox}%
  \begin{beamercolorbox}[wd=.34\paperwidth,ht=2.25ex,dp=1ex,center]{title in head/foot}%
    \usebeamerfont{title in head/foot}
    HiPC 2020
  \end{beamercolorbox}%
  \begin{beamercolorbox}[wd=.33\paperwidth,ht=2.25ex,dp=1ex,right]{date in head/foot}%
    \usebeamerfont{date in head/foot}
    \insertframenumber{} \hspace*{2ex} 
  \end{beamercolorbox}}%
}


\begin{document}

\frame{\titlepage}

\begin{frame}{Summary}

    \begin{itemize}
        \item Nonnegative data can be hierarchically clustered using recursive bipartitioning based on Rank-2 Nonnegative Matrix Factorization (R2-NMF)
        \vfill
        \item Scalable parallelization requires efficient iterative algorithm for R2-NMF as well as data distribution conducive to data-dependent hierarchy structure
        \vfill
        \item Our approach scales well to 1000s of cores and is able to cluster 800GB image dataset
    \end{itemize}
    
\end{frame}

\begin{frame}{Nonnegative Matrix Factorization (NMF) for Clustering}
    \begin{adjustbox}{max totalsize={.4\textwidth}{.5\textheight},center}
    \import{..}{fig/nmf}
    \end{adjustbox}
    \begin{itemize}
        \item approximate $\M{A}$ (features $\times$ samples) into $\M{W}$ (features $\times$ clusters) and $\M{H}$ (samples $\times$ clusters)
        \vfill
        \item nonnegativity gives interpretability of $\M{W}$ and $\M{H}$ as clusters and cluster membership, respectively
        \vfill
        \item applications in text document topic modeling and hyperspectral image segmentation, for example
    \end{itemize}
\end{frame}

\begin{frame}{Hierarchical NMF}

\begin{itemize}
	\item repeatedly use NMF with $k = 2$ to bipartition nodes in order to create a hierarchical tree of clusters
	\item application: hyperspectral imaging
\end{itemize}

\begin{columns}
\begin{column}{.4\textwidth}
    \centering
    \includegraphics[height=0.6\textheight]{../data/DC/0.png}
\end{column}
\begin{column}{.6\textwidth}
	\begin{adjustbox}{width=\textwidth}
    		\import{..}{fig/dc}
   	\end{adjustbox}
\end{column}
\end{columns}

\end{frame}

\begin{frame}{Solving NMF}
    \begin{itemize}
        \item NMF constrained optimization problem: $$\min_{\M{W},\M{H}\geq \M{0}} \|\M{A} - \M{W}\M{H}^\Tra\|_2$$
        
        \vfill
        
        \item Alternating Nonnegative Least Squares (ANLS)
        \begin{itemize}
            \item fix $\M{H}$ and solve the NLS for $\M{W}$ exactly
            \item alternate and repeat until convergence
        \end{itemize}
        
        \vfill

        \item General rank case
        \begin{itemize}
            \item can solve each row exactly using iterative active-set method
            \item determine which values are positive and which are zero
        \end{itemize}
        
        \vfill
        
         \item Rank-2 case
        \begin{itemize}
            \item 4 possibilities: both positive, one or other is zero, both zeroes 
            \item quickly solve each and choose the optimal solution
        \end{itemize}
    \end{itemize}
\end{frame}

\begin{frame}{Parallel R2-NMF}
    \begin{itemize}
        \item Use row distribution of $\M{A}$ and row distributions for $\M{W}$ and $\M{H}$ 
        \item Computational bottlenecks are matrix multiplications
        \begin{itemize}
            \item compute $\M{W}^\Tra\M{A}$ using reduce-scatter for $\M{H}$
            \item compute $\M{A}\M{H}$ using all-gather for $\M{W}$
            \item compute $\M{W}^\Tra\M{W}$ and $\M{H}^\Tra\M{H}$ using all-reduce
        \end{itemize}
    \end{itemize}
        \import{..}{fig/r2nmf}
\end{frame}

\begin{frame}{Bipartitioning with R2-NMF}
    \begin{itemize}
        \item Use columns of $\M{H}^\Tra$ to bipartition the columns of $\M{A}$
        \item Assign the cluster columns into two submatrices
        \item Assign columns of $\M{W}$ as feature signatures of clusters
    \end{itemize}
    \begin{adjustbox}{max totalsize={.7\textwidth}{.6\textheight},center}
        \import{..}{fig/split}
    \end{adjustbox}
\end{frame}

\begin{frame}{Hierarchical Clusterings using R2-NMF (HierNMF2)}
    \begin{itemize}
        \item Split root node into two children with R2-NMF
        \item Choose next node to split based on scoring metric
        \begin{itemize}
        		\item computing score requires pre-computing R2-NMF
	\end{itemize}
        \item Repeat until maximum number of frontier nodes is reached
    \end{itemize}
    \begin{adjustbox}{max totalsize={.7\textwidth}{.6\textheight},center}
        \import{..}{fig/tree}
    \end{adjustbox}
\end{frame}

\begin{frame}{Experimental Data Sets}
    \begin{itemize}
        \item DC-HYDICE (191 $\times$ 392{,}960, 600MB)
        \begin{itemize}
            \item Hyperspectral Digital Imagery Collection Experiment (HYDICE) of the National Mall in Washington, DC
        \end{itemize}
        \vfill
        \item SIIM-ISIC (3{,}145{,}728 $\times$ 33{,}126, 800GB)
        \begin{itemize}
            \item Society for Imaging Informatics in Medicine - International Skin Imaging Colloboration image classification of melanoma images
        \end{itemize}
        \vfill
        \item Synthetic (1{,}048{,}576 $\times$ 11{,}042, 86GB)
        \begin{itemize}
            \item smaller synthetic image classification dataset which has the same aspect ratio as SIIM-ISIC but fits in one node's memory
        \end{itemize}
    \end{itemize}
    \vfill
    \begin{center}
    \scriptsize
    	All experiments performed on Summit at Oak Ridge Leadership Computing Facility
    \end{center}
\end{frame}

\newcommand{\figscal}{.75\textwidth}

\begin{frame}{R2-NMF Strong Scaling and Time Breakdown}
    \centering
    \begin{columns}
        \begin{column}{0.5\textwidth}
            \begin{figure}
            \includegraphics[width=\figscal]{../plots/synthetic_rank2_speedup.pdf}
            \caption{Synthetic Scaling}
            \end{figure}
        \end{column}
        \begin{column}{0.5\textwidth}
            \begin{figure}
            \includegraphics[width=\figscal]{../plots/realworld_rank2_speedup.pdf}
            \caption{\image{} Scaling}
            \end{figure}
        \end{column}
    \end{columns}
\vspace{-.5cm}
    \begin{columns}
        \begin{column}{0.5\textwidth}
            \begin{figure}
            \includegraphics[width=\figscal]{../plots/synthetic_rank2_strongscaling.pdf}
            \caption{Synthetic Breakdown}
            \end{figure}
        \end{column}
        \begin{column}{0.5\textwidth}
            \begin{figure}
            \includegraphics[width=\figscal]{../plots/realworld_rank2_strongscaling.pdf}
            \caption{\image{} Breakdown}
            \end{figure}
        \end{column}
    \end{columns}
    
\begin{itemize}
\small
    \item Nearly perfect strong scaling because time dominated by local matrix multiplication
\end{itemize}
    
\end{frame}

\begin{frame}{Hierarchical Clustering Scaling and Breakdown}
    \centering
    \begin{columns}
        \begin{column}{0.5\textwidth}
            \begin{figure}
            \includegraphics[width=\figscal]{../plots/synthetic_hierarchical_speedup.pdf}
            \caption{Synthetic  Data}
            \end{figure}
        \end{column}
        \begin{column}{0.5\textwidth}
            \begin{figure}
            \includegraphics[width=\figscal]{../plots/realworld_hierarchical_speedup.pdf}
            \caption{\image{} Data}
            \end{figure}
        \end{column}
    \end{columns}
\vspace{-.5cm}
    \begin{columns}
        \begin{column}{0.5\textwidth}
            \begin{figure}
            \includegraphics[width=\figscal]{../plots/synthetic_hier_strongscaling.pdf}
            \caption{Synthetic  Data}
            \end{figure}
        \end{column}
        \begin{column}{0.5\textwidth}
            \begin{figure}
            \includegraphics[width=\figscal]{../plots/realworld_hier_strongscaling.pdf}
            \caption{\image{} Data}
            \end{figure}
        \end{column}
    \end{columns}
    
\begin{itemize}
\small
    \item Scaling hindered by NNLS computations and communication of small subproblems at leaves of hierarchy
\end{itemize}
    
\end{frame}

\begin{frame}{Tree Level Times on Synthetic Data}
    \centering
    \begin{columns}
        \begin{column}{0.5\textwidth}
            \begin{figure}
            \includegraphics[width=\textwidth]{../plots/synthetic_sequential_level_breakdown.pdf}
            \caption{1 Compute Node}
            \end{figure}
        \end{column}
        \begin{column}{0.5\textwidth}
            \begin{figure}
            \includegraphics[width=\textwidth]{../plots/synthetic_parallel_level_breakdown.pdf}
            \caption{40 Compute Nodes}
            \end{figure}
        \end{column}
    \end{columns}

\vfill
    
\begin{itemize}
    \item Evidence that scaling is hindered by leaves of tree
    \begin{itemize}
    	\item MatMul costs are constant over levels of balanced tree 
	\item NNLS and communication costs grow over levels
    \end{itemize}
\end{itemize}
    
\end{frame}

\begin{frame}{Conclusions and Future Directions}

\begin{itemize}
	\item Scalability is possible when MatMul dominates, particularly when input data is tall and skinny and tree is balanced
	\vfill
	\item 2D distribution of matrix $\M{A}$ could improve scalability but requires redistribution of $\M{A}$ for subtrees
	\vfill
	\item Parallelization across tree could improve scalability but requires careful load balancing and decentralizing split orders
	\vfill
	\item Hierarchical clustering can also be used to initialize flat NMF, particularly when $k$ is large, for Divide-and-Conquer NMF
\end{itemize}

\end{frame}



\begin{frame}
    \frametitle{Author Contact Information}
    
    For questions, please contact
    \begin{itemize}
        \item Lawton Manning: \email{mannlg15@wfu.edu}
        \item Grey Ballard: \email{ballard@wfu.edu}
        \item Ramakrishnan Kannan: \email{kannanr@ornl.gov}
        \item Haesun Park: \email{hpark@cc.gatech.edu}
    \end{itemize}
    
\end{frame}
    

\end{document}