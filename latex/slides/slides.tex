\documentclass{beamer}

\usepackage{cite,doi}
\usepackage{amsmath,amssymb,amsfonts}
\usepackage{algorithm,algpseudocode}
\usepackage{graphicx}
\usepackage{textcomp}
\usepackage{xcolor}
\usepackage{cleveref}
\usepackage{tikz}
\usepackage{mathtools}
\usepackage{../ourmacros}
\usepackage{subcaption}
\usepackage{import}
\usepackage{adjustbox}

\newcommand{\hyper}{DC-HYDICE}
\newcommand{\image}{SIIM-ISIC}

\definecolor{wfugold}{rgb}{0.6196078,0.494117647,0.21960784}
\newcommand{\red}[1]{\textcolor{red}{#1}}
\newcommand{\blue}[1]{\textcolor{blue}{#1}}
\newcommand{\multiplycolor}{red}
\newcommand{\zero}{}
\newcommand{\cred}[1]{\textcolor{red}{#1}}
\newcommand{\cblue}[1]{\textcolor{blue}{#1}}
\newcommand{\cgold}[1]{\textcolor{wfugold}{#1}}
\newcommand{\email}[1]{\href{mailto:#1}{\texttt{#1}}}

\title{Parallel Hierarchical Clustering using \\ Rank-Two Nonnegative Matrix Factorization}

\author{
    Lawton Manning\inst{1}
    \and Grey Ballard\inst{1}\\
    \and Ramakrishnan Kannan\inst{2}
    \and Haesun Park\inst{3}
}

\institute{
    \inst{1}%
    Wake Forest University
    \and
    \inst{2}%
    Oak Ridge National Laboratory
    \and
    \inst{3}%
    Georgia Institute of Technology
}

\date{
    27th IEEE International Conference on High Performance Computing, Data, \& Analytics (HiPC 2020)
}

\usetheme{Warsaw}
\usecolortheme[named=wfugold]{structure}
%Madrid
%\usecolortheme{dolphin}
%\useinnertheme{rounded}
%\usefonttheme{serif}
\setbeamertemplate{navigation symbols}{} % gets rid of navigation bars
\setbeamertemplate{footline}
{
  \hbox{%
  \begin{beamercolorbox}[wd=.33\paperwidth,ht=2.25ex,dp=1ex,left]{author in head/foot}%
    \usebeamerfont{author in head/foot}
    Manning, Ballard, Kannan, Park
  \end{beamercolorbox}%
  \begin{beamercolorbox}[wd=.34\paperwidth,ht=2.25ex,dp=1ex,center]{title in head/foot}%
    \usebeamerfont{title in head/foot}
    HiPC 2020
  \end{beamercolorbox}%
  \begin{beamercolorbox}[wd=.33\paperwidth,ht=2.25ex,dp=1ex,right]{date in head/foot}%
    \usebeamerfont{date in head/foot}
    \insertframenumber{} \hspace*{2ex} 
  \end{beamercolorbox}}%
}


\begin{document}

\frame{\titlepage}

\begin{frame}{Nonnegative Matrix Factorization (NMF)}
    \begin{itemize}
        \item $\M{A} \approx \M{W}\M{H}^\Tra$
        \item approximate $\M{A}$ (features $\times$ samples) into $\M{W}$ (features $\times k$) and $\M{H}$ (samples $\times k$)
        \item nonnegativity gives interpretability of $\M{W}$ and $\M{H}$ as clusters and cluster membership, respectively
        \item applications
            \begin{itemize}
                \item text document clustering using bag of words or TF-IDF matrices
                \item hyperspectral imaging segmentation
            \end{itemize}
    \end{itemize}
\end{frame}

\begin{frame}{Hierarchical NMF}
    \begin{itemize}
        \item repeatedly use NMF with $k = 2$ to create a hierarchical tree of clusters
        \item application: hyperspectral imaging
    \end{itemize}
    \begin{adjustbox}{max totalsize={.7\textwidth}{.6\textheight},center}
    \import{..}{fig/dc}
    \end{adjustbox}
\end{frame}

\begin{frame}{Solving NMF}
    \begin{itemize}
        \item constrained optimization: $$\min_{\M{W},\M{H}\geq \M{0}} \|\M{A} - \M{W}\M{H}^\Tra\|_2$$
        \item alternating update
        \begin{itemize}
            \item fix $\M{H}$ and solve the NNLS for $\M{W}$ exactly
            \item alternate and repeat until convergence
        \end{itemize}
        \item NNLS Solvers
        \begin{itemize}
            \item Block Principal Pivoting (BPP)
            \item ??? Other Method ???
        \end{itemize}
    \end{itemize}
\end{frame}

\begin{frame}{Rank-2 NMF (R2NMF)}
    \begin{itemize}
        \item when $k = 2$, BPP can be solved quickly as the size of the active set is 4
        \item Active Sets for R2NMF (row by row)
        \begin{itemize}
            \item both columns nonnegative
            \item only left column nonnegative
            \item only right column nonnegative
            \item both columns negative
        \end{itemize}
    \end{itemize}
\end{frame}

\begin{frame}{Parallel R2NMF}
\end{frame}

\begin{frame}{Splitting with R2NMF}
    \begin{itemize}
        \item use columns of $\M{H}^\Tra$ to split the columns of $\M{A}$ into two submatrices
        \item assign the corresponding columns of $\M{W}$ to the submatrices
        \item repeat NMF and split on submatrices
    \end{itemize}
    \begin{adjustbox}{max totalsize={.7\textwidth}{.6\textheight},center}
        \import{..}{fig/split}
    \end{adjustbox}
\end{frame}

\begin{frame}{HierNMF}
    \begin{itemize}
        \item call R2NMF on root node to split into two children
        \item use the power iteration to select children that least approximate rank-1 matrices
        \item repeat until maximum number of nodes is reached or all child matrices are rank-1
    \end{itemize}
    \begin{adjustbox}{max totalsize={.7\textwidth}{.6\textheight},center}
        \import{..}{fig/tree}
    \end{adjustbox}
\end{frame}

\begin{frame}{Parallel HierNMF}
    FIXME: Need data distribution figure
\end{frame}

\begin{frame}{Data Sets}
    \begin{itemize}
        \item DC-HYDICE
        \begin{itemize}
            \item Hyperspectral Digital Imagery Collection Experiment (HYDICE) of the National Mall in Washington, DC
        \end{itemize}
        \item SIIM-ISIC
        \begin{itemize}
            \item Society for Imaging Informatics in Medicine - International Skin Imaging Colloboration image classification of melanoma images
        \end{itemize}
        \item Synthetic Image classification
        \begin{itemize}
            \item smaller image classification dataset which has the same aspect ratio as SIIM-ISIC but small enough to fit in memory on one machine
        \end{itemize}
    \end{itemize}
\end{frame}

\begin{frame}
\frametitle{Author Contact Information}

For questions, please contact
\begin{itemize}
	\item Lawton Manning: \email{mannlg15@wfu.edu}
	\item Grey Ballard: \email{ballard@wfu.edu}
	\item Ramakrishnan Kannan: \email{kannanr@ornl.gov}
	\item Haesun Park: \email{hpark@cc.gatech.edu}
\end{itemize}

\end{frame}

\end{document}