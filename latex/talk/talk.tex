\documentclass{beamer}

\usepackage{cite,doi}
\usepackage{amsmath,amssymb,amsfonts}
\usepackage{algorithm,algpseudocode}
\usepackage{stackengine,graphicx}
\usepackage{textcomp}
\usepackage{xcolor}
\usepackage{cleveref}
\usepackage{tikz}
\usetikzlibrary{3d}
\usetikzlibrary{patterns}
\usetikzlibrary{calc}
\usetikzlibrary{arrows}
\usetikzlibrary{matrix}
\usetikzlibrary{positioning}
\usetikzlibrary{decorations.pathreplacing}
\usepackage{mathtools}
\usepackage{subcaption}
\usepackage{import}
\usepackage{bm}
\usepackage[mathscr]{eucal}
\usepackage{amsbsy}
\usepackage[section]{placeins}
\usepackage{adjustbox}
\stackMath

\usepackage{../ourmacros}

\newcommand{\T}[1]{\boldsymbol{\mathscr{#1}}}

\newcommand{\GB}[1]{\textcolor{red}{\textbf{GB}: #1}}
\newcommand{\hyper}{DC-HYDICE}
\newcommand{\image}{SIIM-ISIC}

\definecolor{wfugold}{rgb}{0.6196078,0.494117647,0.21960784}
\newcommand{\red}[1]{\textcolor{red}{#1}}
\newcommand{\blue}[1]{\textcolor{blue}{#1}}
\newcommand{\multiplycolor}{red}
\newcommand{\zero}{}
\newcommand{\cred}[1]{\textcolor{red}{#1}}
\newcommand{\cblue}[1]{\textcolor{blue}{#1}}
\newcommand{\cgold}[1]{\textcolor{wfugold}{#1}}
\newcommand{\email}[1]{\href{mailto:#1}{\texttt{#1}}}


\newlength\matfield
\newlength\tmplength
\def\matscale{1.}
\newcommand{\dimbox}[3]{%
  \setlength\matfield{\matscale\baselineskip}%
  \setbox0=\hbox{\vphantom{X}\smash{#3}}%
  \setlength{\tmplength}{#1\matfield-\ht0-\dp0}%
  \fboxrule=1pt\fboxsep=-\fboxrule\relax%
  \fbox{\makebox[#2\matfield]{\addstackgap[.5\tmplength]{\box0}}}%
}
\newcommand{\raiserows}[2]{%
   \setlength\matfield{\matscale\baselineskip}%
   \raisebox{#1\matfield}{#2}%
}
\newcommand{\matbox}[5]{
  \stackunder{\dimbox{#1}{#2}{$#5$}}{\scriptstyle(#3\times #4)}%
}

\title{Parallel Hierarchical Clustering using \\ Rank-Two Nonnegative Matrix Factorization}

\author{
    Lawton Manning\inst{1}
    \and Grey Ballard\inst{1}\\
    \and Ramakrishnan Kannan\inst{2}
    \and Haesun Park\inst{3}
}

\institute{
    \inst{1}%
    Wake Forest University
    \and
    \inst{2}%
    Oak Ridge National Laboratory
    \and
    \inst{3}%
    Georgia Institute of Technology
}

\date{
    27th IEEE International Conference on High Performance Computing, Data, \& Analytics (HiPC 2020)
}

\usetheme{Warsaw}
\usecolortheme[named=wfugold]{structure}
%Madrid
%\usecolortheme{dolphin}
%\useinnertheme{rounded}
%\usefonttheme{serif}
\setbeamertemplate{navigation symbols}{} % gets rid of navigation bars
\setbeamertemplate{footline}
{
  \hbox{%
  \begin{beamercolorbox}[wd=.33\paperwidth,ht=2.25ex,dp=1ex,left]{author in head/foot}%
    \usebeamerfont{author in head/foot}
    Manning, Ballard, Kannan, Park
  \end{beamercolorbox}%
  \begin{beamercolorbox}[wd=.34\paperwidth,ht=2.25ex,dp=1ex,center]{title in head/foot}%
    \usebeamerfont{title in head/foot}
    HiPC 2020
  \end{beamercolorbox}%
  \begin{beamercolorbox}[wd=.33\paperwidth,ht=2.25ex,dp=1ex,right]{date in head/foot}%
    \usebeamerfont{date in head/foot}
    \insertframenumber{} \hspace*{2ex} 
  \end{beamercolorbox}}%
}
\usepackage{caption}
\captionsetup[figure]{labelformat=empty}%


\begin{document}

\frame{\titlepage}

\begin{frame}{Nonnegative Matrix Factorization (NMF)}
    \begin{adjustbox}{max totalsize={.4\textwidth}{.5\textheight},center}
        \import{..}{fig/nmf}
    \end{adjustbox}
    \begin{itemize}
        \item Approximate $\M{A}$ (features $\times$ samples) into:
        \begin{itemize}
            \item $\M{W}$ (features $\times$ clusters): feature signatures
            \item $\M{H}$ (samples $\times$ clusters): cluster membership
        \end{itemize}
        \vfill
        \item Clustering applications
        \begin{itemize}
            \item text document topic modeling
            \item hyperspectral image segmentation
        \end{itemize}
    \end{itemize}
\end{frame}

\begin{frame}{Hierarchical Clustering}
    \begin{columns}
        \begin{column}{.4\textwidth}
            \begin{itemize}
                \item Recursively bipartition data to form binary tree
                \vfill
                \item Decision choices
                \begin{itemize}
                    \item how to split
                    \begin{itemize}
                        \item Rank-2 NMF
                    \end{itemize}
                    \item choosing node to split
                    \begin{itemize}
                        \item based on Rank-2 NMF
                    \end{itemize}
                \end{itemize}
            \end{itemize}
        \end{column}
        \begin{column}{0.6\textwidth}
            \begin{adjustbox}{width=\textwidth}
                \import{..}{fig/dc_labels}
           \end{adjustbox}
        \end{column}
    \end{columns}
\end{frame}

\begin{frame}{Splitting with Rank-2 NMF}
    \begin{itemize}
        \item Use columns of $\M{H}^\Tra$ to bipartition the columns of $\M{A}$
        \item Assign the cluster columns into two submatrices
        \item Assign columns of $\M{W}$ as feature signatures of clusters
    \end{itemize}
    \begin{adjustbox}{max totalsize={.7\textwidth}{.6\textheight},center}
        \import{..}{fig/split}
    \end{adjustbox}
\end{frame}

\begin{frame}{Solving NMF}
    $$\min_{\M{W},\M{H}\geq \M{0}} \|\M{A} - \M{W}\M{H}^\Tra\|_2$$
    \begin{itemize}
        \item Alternating Nonnegative Least Squares (ANLS)
        \begin{itemize}
            \item fix H, solve for W using NLS
            \item alternate and repeat
        \end{itemize}
        \vfill
        \item Rank-2 NLS
        \begin{itemize}
            \item optimized version of general rank-$k$ algorithm
            \item 4 possibilities: ++,+0,0+,00
            \item solve each and choose optimal
        \end{itemize}
    \end{itemize}
\end{frame}

\begin{frame}{Parallel Rank-2 NMF}
    \begin{itemize}
        \item Use row distribution of $\M{A}$, $\M{W}$, and $\M{H}$ 
        \begin{itemize}
        		\item example distribution across 3 processors shown below
        \end{itemize}
        \vfill
        \item Computational bottlenecks are matrix multiplications
        \begin{itemize}
            \item compute $\M{W}^\Tra\M{A}$ using reduce-scatter for $\M{H}$
            \item compute $\M{A}\M{H}$ using all-gather for $\M{W}$
            \item compute $\M{W}^\Tra\M{W}$ and $\M{H}^\Tra\M{H}$ using all-reduce
        \end{itemize}
    \end{itemize}
    \import{..}{fig/r2nmf}
\end{frame}


\newcommand{\figscal}{\textwidth}

\begin{frame}{Hierarchical Clustering Scaling and Breakdown}
    \centering
    \begin{columns}
        \begin{column}{0.5\textwidth}
            \begin{figure}
            \includegraphics[width=\figscal]{../plots/realworld_hierarchical_speedup.pdf}
            \caption{\image{} Data}
            \end{figure}
        \end{column}
        \begin{column}{0.5\textwidth}
            \begin{figure}
            \includegraphics[width=\figscal]{../plots/realworld_hier_strongscaling.pdf}
            \caption{\image{} Data}
            \end{figure}
        \end{column}
    \end{columns}
    \begin{itemize}
    \small
        \item SIIM-ISIC (3{,}145{,}728 $\times$ 33{,}126, 800GB)
        \begin{itemize}
            \item Society for Imaging Informatics in Medicine - International Skin Imaging Colloboration image classification of melanoma images
        \end{itemize}
        \item Reasonable scaling that is dominated by MatMul
    \end{itemize}
\end{frame}

\begin{frame}{Conclusions and Future Directions}

\begin{itemize}
	\item Scalability is possible when MatMul dominates, particularly when input data is tall and skinny and tree is balanced
	\vfill
	\item Possibilities for improving scalability
	\begin{itemize}
		\item general 2D data distribution
		\item parallelization across hierarchical tree
	\end{itemize}
	\vfill
	\item HierNMF2 as a preprocessing step for Flat NMF
\end{itemize}

\end{frame}

\end{document}