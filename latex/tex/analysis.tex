% !TEX root = ../paper.tex

\subsubsection{Analysis}
\label{sec:analysis}

\paragraph{\emph{Parallel Rank-2 NMF}}

Each iteration of \Cref{alg:parrank2nmf} incurs the same cost, so we analyze per-iteration computation and communication costs.
We first consider the cost of the Rank-2 NNLS solves, which are local computations.
In the notation of \Cref{alg:r2nnls}, matrix $\M{G}$ is $2\times 2$, so solving the unconstrained system (via Cholesky decomposition) and then choosing between single-positive-variables solutions if necessary requires constant time per row of $\M{C}$.
Thus, the cost of \Cref{alg:r2nnls} is proportional to the number of rows of the first input matrix.
In the context of \Cref{alg:parrank2nmf}, the per-iteration computational cost of rank-2 solves is then $O((m+n)/p)$.
The other local computations are the matrix multiplications $\M[\hat]{W}^\Tra \M[\hat]{W}$ and $\M[\hat]{H}^\Tra \M[\hat]{H}$, which also amount to $O((m+n)/p)$ flops, and $\M[\hat]{A}^\Tra \M[\hat]{W}$ and $\M[\hat]{A} \M{H}$, which require $O(mn/p)$ flops because they involve the data matrix.
Thus, the computation cost is $\gamma \cdot O((mn+m+n)/p)$ and typically dominated by the multiplications involving $\M{A}$.
We track the lower order terms corresponding to NNLS solves because their hidden constants are larger than that of the dominating term.

There are four communication collectives each iteration, and each involves all $p$ processors.
The two all-reduce collectives to compute the Gram matrices of the factor matrices involve $2\times 2$ matrices and incur a communication cost of $(\gamma+\beta + \alpha) \cdot O(\log p)$.
The reduce-scatter and all-gather collectives involve $n\times 2$ matrices (the size of $\M{H}$) and require $\beta \cdot O(n) + \alpha \cdot O(\log p)$ in communication cost (we ignore the computation cost of the reduce-scatter because it is typically dominated by the bandwidth cost).
If the algorithm performs $\imath$ iterations, the overall cost of \Cref{alg:parrank2nmf} is
\begin{equation}
\label{eq:r2nmfcost}
\gamma \cdot O\left( \frac{\imath (mn+m+n)}{p} \right) + \beta \cdot O(\imath n) + \alpha \cdot O(\imath \log p).
\end{equation}

\paragraph{\emph{Parallel Power Method}}

Similar to the previous analysis, we consider a single iteration of the power method.
The local computation is dominated by two matrix-vector products involving the local data matrix of size $O(mn/p)$ words, incurring $O(mn/p)$ flops.
The single communication collective is an all-reduce of the approximate right singular vector, which is of size $n$, incurring $\beta \cdot O(n) + \alpha \cdot O(\log p)$ communication.
We ignore the $O(n)$ computation cost of normalizing the vector, as it will typically be dominated by the communication cost of the all-reduce.
Over $\jmath$ iterations, \Cref{alg:parpowmeth} has an overall cost of
\begin{equation}
\label{eq:powmethcost}
\gamma \cdot O\left( \frac{\jmath mn}{p} \right) + \beta \cdot O(\jmath n) + \alpha \cdot O(\jmath \log p).
\end{equation}
Note the per-iteration cost of the power method differs by only a constant from the per-iteration cost of Rank-2 NMF.
Because the power method involves single vectors rather than factor matrices with two columns, its constants are smaller than half the size of their counterparts.

\paragraph{\emph{Hierarchical Clustering}}

To analyze the overall cost of the hierarchical clustering algorithm, we sum the costs over all nodes in the tree.
Because the shape of the tree is data dependent and affects the overall costs, for the sake of analysis we will analyze only complete levels.
The number of rows in any node is $m$, the same as the root node, as each splitting corresponds to a partition of the columns.
Furthermore, because each split is a partition, every column of $\M{A}$ is represented exactly once in every complete level of the tree.
If we assume that all nodes perform the same number of NMF iterations ($\imath$) and power method iterations $(\jmath)$, then the dominating costs of a node with $\overline{n}$ columns is
\begin{equation*}
\label{eq:nodecost}
\begin{split}
\gamma \cdot O\left( \frac{(\imath+\jmath) m\overline{n} + \imath (m+\overline{n}) }{p} \right) + \beta \cdot O((\imath+\jmath) \overline{n}) \\ + \alpha \cdot O((\imath+\jmath) \log p).
\end{split}
\end{equation*}
Because the sum of the number of columns across any level of the tree is $n$, the cost of the $\ell$th level of the tree is
\begin{equation}
\label{eq:levelcost}
\begin{split}
\gamma \cdot O\left( \frac{(\imath+\jmath) mn + \imath m 2^\ell}{p} \right) + \beta \cdot O((\imath+\jmath) n) \\ + \alpha \cdot O((\imath+\jmath) 2^\ell \log p).
\end{split}
\end{equation}
Note that the only costs that depend on the level index $\ell$ are the latency cost and a lower-order computational cost.

Summing over levels and assuming the tree is nearly balanced and has height $O(\log k)$ where $k$ is the number of frontier nodes, we obtain an overall cost of \Cref{alg:hiernmf2} of
\begin{equation}
\label{eq:treecost}
\begin{split}
\gamma \cdot O\left( \frac{(\imath+\jmath) mn}{p}  \log k + \frac{\imath mk}{p} \right) + \beta \cdot O((\imath+\jmath) n \log k) \\ + \alpha \cdot O((\imath+\jmath) k \log p).
\end{split}
\end{equation}

We see that the leading order computational cost is logarithmic in $k$ and perfectly load balanced.
If the overall running time is dominated by the computation (and in particular the matrix multiplications involving $\M{A}$), we expect near-perfect strong scaling.
The bandwidth cost is also logarithmic in $k$ but does not scale with the number of processors.
The latency cost grows most quickly with the target number of clusters $k$ but is also independent of the matrix dimensions $m$ and $n$.
