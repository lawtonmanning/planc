% !TEX root = ../paper.tex

\section{Conclusion}

As shown in the theoretical analysis (\Cref{sec:analysis}) and experimental results (\Cref{sec:hiernmf2scaling}), \Cref{alg:hiernmf2} can efficiently scale to large $p$ as long as the execution time is dominated by local matrix multiplication.
The principal barriers to scalability are the bandwidth cost due to Rank-2 NMF, which is consistent across levels of the tree and proportional to the number of columns $n$ of the original data set, and the latency cost due to large numbers of tree nodes in lower levels of the tree.
When $n$ is small relative to $m$ and the number of leaves $k$ and levels $\ell$ are small, then these barriers do not pose a problem until $p$ is very large.
However, if the input matrix is short and fat (i.e., has many samples with few features), then the bandwidth cost can hinder performance for smaller $p$.
Likewise, if $k$ is large or the tree is lopsided, then achieving scalability for very small problems is more difficult.
We also note that in the case of sparse $\M{A}$, it becomes more difficult to hide communication behind the cheaper matrix multiplications, and other costs may become more dominant.

One approach for reducing the bandwidth cost of Rank-2 NMF is to choose a more balanced data distribution over a 2D grid, as proposed by Kannan et al. \cite{KBP16}.
This will reduce the communicated data and achieve a local data matrix that is more square, which can improve local matrix multiplication performance.
The downside of this approach is that it requires a redistribution of the data for each split, but if many NMF iterations are required, then the single upfront redistribution cost may be amortized.

Another approach to alleviate the rising latency costs of lower levels of the tree is to parallelize across nodes of the tree.
This will result in fewer processors working on any given node, reducing the synchronization time among them, and it will allow small, latency-bound problems to be solved simultaneously.
Prioritizing the sequence of node splits is more difficult in this case, but modifying the stopping criterion for splitting to use a score threshold instead of a target number of leaves will allow truly independent computation.

In the future, we also plan to compare performance of \Cref{alg:hiernmf2} with flat NMF algorithms and employ the Divide-and-Conquer NMF technique \cite{DKDP17} of seeding an iterative flat NMF algorithm with the feature vectors of the leaf nodes.
The parallel technique proposed here can be combined with the existing PLANC library \cite{EH+19-TR} to obtain faster overall convergence for very large datasets.
